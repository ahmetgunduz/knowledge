\documentclass[a4paper]{report}
\usepackage{verbatim}
\begin{document}

\author{Florian Kaufmann}
\title{}
\date{\today}
\maketitle

\begin{abstract}
\end{abstract}

\tableofcontents

\section{Comments}

\subsection{Creating & killing}

Creating comments out of thin air: comment-dwim (region not active)
creates a new comment. If point is on an empty line, the new comment is
aligned to the code. If prefix arg is specified, insert that many
comment chars. If point is on an non-empty line, calls comment-indent
which on a line without comment inserts a new comment.

Commenting out regions: comment-dwim (region active) calls
comment-region, which comments out using prefix arg comment characters;
negative means delete that many. C-u comment-out-regions let's you
specify the comment start delimiter explicitely.

Removing comments: C-u [arg] comment-dwim (region not active) on a line
with at least one comment calls kill-comment. kill-comment does kill the
first comment on the next prefix arg lines. C-u comment-dwim (region
active) removes the comments if the whole region is commented out. C-u
comment-region removes one comment char. comment-region with neg prefix
removes that many comment chars.

\subsection{Editing existing}

With auto-fill-mode turned on, simply type. Text is broken at
current-fill-column. [???]

Add new comment line: comment-indent-new-line like new-line-and-indent,
comment delimiters remain ok. This command is intended for styles where
you write a comment per line, starting a new comment (and terminating it
if necessary) on each line. See comment-multi-line, comment-continue.

c-indent-new-comment-line

- newline-and-indent: ??

To jump into the comment of the current line, use comment-dwim - which
however at the same time also indents the comment.

\subsection{Options}

New comments are created using comment-start and comment-end. Some 
commands can insert multiple comment chars. The char is the last
nonwhite of comment-start. In case of commenting out regions
comment-style determines various additional stuff. comment-continue is
used instead comment-start on multiline comments (commenting out
multiple lines, or continuing a comment). If nil, the its value is
derived from comment-start by replacing its first char with a space.
comment-padding 

To find comments, e.g. to remove them or to tell if there is a comment on
the line, the regexps comment-start-skip and comment-end-skip are
used. 

if fill-prefix is given, it is as prefix before ??comment-start?? for
comment-indent-new-line, comment-indent, but not for comment-region.


comment-start + add*char_start + padding
padding + add*char_end + comment-end

Options/Variables
- comment-column: ToEOL begin here
- c-indent-comment-alist: adaptively (depending on current line)
determine value instead of comment-column
- c-indent-comments-syntactically-p: ???
- set-comment-column
- comment-(start|end): For newly created comments [??? how to define ToEOF/Multiline?]
- comment-(start|end)-skip: Regexp describing start/end of a comment
upto its body. Probably used internaly for recognizing start/end of
existing comments.
- comment-continue: prefix for multiline comment. ??? Difference to all the
fill prefixes?? It seems like fill-prefix is used if its non-nil.

- comment-padding: for comment-region: spaces between text and comment
  delimiters, or string to be inserted
- comment-style: how comments look like created by comment-region 

- comment-indent-function: to compute indentation for a comment
- c-comment-prefix-regexp: ?? Determine line prefix when a new comment
  line is made.
- c-block-comment-prefix: Comment prefix when a \em{block} comment is broken
  for the first time; because then there is no 2nd+ line from where the
  prefix can be deduced.
- comment-multi-line: ?
- comment-fill-column replaces fill-column within comments
- comment-empty-lines: comment-region does not comment empty lines.  

\begin{verbatim}

  // test  
+ 111     2                  
  int i;                          

insertions:        
strings        
1 comment-start                
2 comment-end
paddings
a            
recognitions:        
\end{verbatim}

\subsection{Filling}
See main chapter filling.

\subsection{Thoughts}

Delimitertype
  - ToEOF
  - Multiline

Content  
  - PlainComment
  - Documentation (POD,doxygen?,...)
  
Usual places
  - End of code line, aligned with other comments
  - On line bevore code, aligned with code
  - Within code 

- http://aurelien.tisne.free.fr/emacs-pages/emacs/prg-comment-doc.html
- any template modes


\subsection{wants}
- dynamic comment-column
- yank/kill comment text with filladapt's prefix/postfix removed




\section{Skelletons/Templates}

\section{Simply type}

- Auto fill mode: auto newlines so text is always nicely aligned. Not
  within code controlled by cc mode, see c-ignore-auto-fill.
  ??? How to continue end-of-line-comments in general??
  
- cc mode / auto-newline minor mode: auto newlines when pressing certain
  electric keys. electric minor mode seems to be a prequisite.

- c-context-open-line  

  
\section{Beautify}

\subsection{Misc}
align: 

\subsection{Filling}

Adaptive fill mode enhances Emacs auto fill mode. Key improvment is that
fill-prefix is dynamically determined. Also the boundaries of the
current paragraph are located.

?? Distinguish indendation of first line of paragraph to others
?? EOL comment over multiple lines

See also auto fill

Variables:
- c-ignore-auto-fill: list of contexts (e.g. string, ... ) where auto
  fill mode is disabled
- auto-fill-function: 
- normal-auto-fill-function: auto-fill-function is initialized (when?)
- comment-fill-column replaces fill-column within comments
with it.

\section{Simply write}

\subsection{newlines/indenting}

- Auto fill mode: is a minor mode in which lines are broken automatically
when they become too long, and the text is nicely aligned.
- newline-and-indent:
- indent-new-comment-line: Alias for comment-indent-new-line 
- comment-indent-new-line: ??
- c-context-line-break: 
- c-context-open-line:
- c-indent-new-comment-line:

\subsection{misc}
- cc mode / c hungry delete mode
  - c-electric-(backspace|delete-forward):
  - c-hungry-delete-(backwards|forwards): As above, but doesn't need
    minor mode; is usaly on own binding (i.e. not backspace,C-d)

\section{Completion}
cyclebuffer
ido
joc-toggle-buffer
lcomp (list-completion hacks)
minibuffer-complete-cycle
helm
ff-paths
emacs-ycmd

\section{IDE}
- http://tuhdo.github.io/c-ide.html
- https://github.com/Sarcasm/irony-mode with company-irony
- helm
- emacs-ycmd

\section{Logical Hierarchy}

\subsection{Outline mode}
\begin{verbatim}

leave = body

e/c : show/hide current's headings body 
s/d : show/hide subtree 
 /l :     /hide subtree's bodies 
 /t :     /hide all bodies
a   : show/     everything

tab  show children = show all direct subheadings
k    show branches = show all subheadings, no bodies
s    show subtree  = show everything

l		 hide leaves = hide all bodies/leaves
d    hide subtree = everything
\end{verbatim}
camel casse minor mode:
cc mod / subword minor mode: as camel case mode

\section{(book)marks}

\subchapter{wants}
- visualize (book)marks in buffer. Within buffer and/or at fringe.

\section{window/frames}

\section{help}
sys-apropos

\section{other window}

display-buffer::
switch-to-buffer::
\... When the selected window is the minibuffer window or is strongly
dedicated to its buffer (*note Dedicated Windows::), this function calls
`pop-to-buffer'. ...
pop-to-buffer(buffer-or-name &optional other-window norecord)::
This command makes BUFFER-OR-NAME the current buffer and switches
to it in some window, preferably not the window previously
selected. 
\... to be read ... !!!!
window-dedicated-p::
next-error::
(if from-compilation-buffer
        ;; If the compilation buffer window was selected,
        ;; keep the compilation buffer in this window;
        ;; display the source in another window.
        (let ((pop-up-windows t))
          (pop-to-buffer (marker-buffer mk) 'other-window))
      (if (window-dedicated-p (selected-window))
          (pop-to-buffer (marker-buffer mk))
        (switch-to-buffer (marker-buffer mk))))




pop-up-windows::
  Non-nil means `display-buffer' should make a new window.
compilation-goto-locus::
display-buffer-reuse-frames::

special-display-regexps::
special-display-buffer-names::
If regexp/name matches.... By default, special display means to give the
buffer a dedicated frame. See `special-display-function'.

same-window-regexps::
same-window-buffer-names::
\... If the buffer's name is in this list, `display-buffer' handles the buffer by
switching to it in the selected window.

special-display-function::
This variable holds the function to call to display a buffer
specially.  It receives the buffer as an argument, and should
return the window in which it is displayed.  The default value of
this variable is `special-display-popup-frame', see below.


pop-up-frames::
Whether `display-buffer' should make a separate frame.
If nil, never make a seperate frame.
If the value is `graphic-only', make a separate frame
on graphic displays only.
Any other non-nil value means always make a separate frame.
  


what happens when file/buffer is visted (topics: variables, hooks, encoding,
major mode, minor mode)

- dir local variables
- file local variables
  - mode from file local variables
  - encoding from file local variables
- auto-mode-alist
- magic-mode-alist
- interpreter-mode-alist
- auto-mode-alist
- change-major-mode-after-body-hook
- mymode-hook (for each base mode up to root)
- after-change-major-mode-hook


How coding of a file is determined: see info pages on find-auto-coding:
... The order of looking for a matching rule is ‘auto-coding-alist’ first,
then ‘auto-coding-regexp-alist’, then the ‘coding:’ tag, and lastly
‘auto-coding-functions’.  If no matching rule was found, the function returns
‘nil’. [When does file-coding-system-alist come?]

\end{document}
