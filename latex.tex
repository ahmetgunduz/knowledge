\documentclass[a4paper]{report}

\begin{document}

\author{Florian Kaufmann}
\title{Latex}
\maketitle

\tableofcontents

\section{Introduction}

\subsection{History}

\TeX is a typesetting system that was designed and mostly written in the
 years 1978-83 by Donald E. Knuth. \TeX is spelled ``tech''. 
 
\LaTeX is a document markup language and document preparation system
setting up on \Tex{} and was designed 1982-85 by Leslie Lamport.

Advantages of \LaTeX

\begin{itemize}
 \item Its free
 \item All versions are backward compatible. You will virtually be able
       to open a document in eternety.
 \item Based on textfiles - so the source can be stored in a version
       control system, such as subversion.
 \item Supported by an incredible wide range of computer systems.
 \item Output looks good because it is consistent over the whole (large)
       document and because the look was made by professionals, not by
       amateurs as in an typicall WYSIWYG document.
 \item One can concentrate on the content and doesn't has to care about
       the form.
\end{itemize}

\subsection{Basic document structure}

\begin{verbatim}
document = 
  preamble
  \begin{document}
  documenttext
  \end{document}; # follwing stuff is ignored
\end{verbatim}

\subsection{Character set}

\LaTeX is based on the ASCII char set. Thus special characters can not
directly be given.

Some characters have a special meaning to \LaTeX. The full set of
those meta characters together with their representation is listed in
the following.

\begin{verbatim}
$  \$       ~  \~{}
&  \&       ^  \^{}
%  \%       "  \dq 
#  \#       \  \textbacksslash 
_  \_       |  \textbar        
{  \{       <  \textless       
}  \}       >  \textgreater    
\end{verbatim}

\subsection{Special punctuation characters}

\begin{verbatim}
hyphen: -: daughter-in-law
en-dash: --: pages 13--67
em-dash: ---: yes---or no?
minus-sign: $-1$

????: \~       # uperscript tilde
tilde: \~{}    # 
tilde: $\sim$  # bigger, more in the middle

Ellipsis: \ldots # ...
\end{verbatim}

\subsection{Comments}

\begin{verbatim}
% Comment to end of line
\begin{comment}
multilinecomment
\end{comment}
\end{verbatim} 

\subsection{commands}

Commandnames are case sensitive. 

\begin{verbatim}
Command = 
  \\(\W|\w+)
  [\*]
  [\[ (arg \,{f~} )+ \]]      ## optional arguments
  [\{ (arg \,{f~} ){-,9} \}]; ## arguments
\end{verbatim}

In case of commands where the command name consists of letters and where
there are no arguments, the command name, i.e. the command, is ended by
the next non-letter character. For example a space, which however is
then eaten up. To insert a space after that command, either use two
spaces, or use \verb|\ |. Another way is to explicetely pass zero
arguments like that \verb|\mycommand{}|.


\subsection{Spaces, newlines}

A period following an uppercase character is not taken as the end of a
sentence. Use command $\backslash$@ before a period to tell Latex that that
period is not the end of the sentence.

\subsection{Structure}

Paragagraphs: A new paragraph is started by either two or more empty
lines or by the command \verb|\par|. 

\section{Syntax}

\begin{verbatim}
Command = \\CmdName( \{ Arg \} | \[ Arg \] )+

CmdName = (Letter+|^Letter)        
        
Letter = '[a-zA-Z];        
        
Name = 
\end{verbatim}

\section{Math}

\begin{verbatim}
\begin{math} ... \end{math}
$ ... $
\( ... \)

\begin{displaymath} ... \end{displaymath}
$$ ... $$
\[ ... \]

\begin{equation} ... \end{equation}

\textrm{ ... }
\end{verbatim}
  
Building blocks

\begin{tabular}
\verb|\lim_{l}| & Limes \\
\verb|\sum_{from}^{to}| & Sum operator\\
\verb|\int_{from}^{to}| & Integral \\
\verb|\prod_{from}^{to}| & Product operator \\
\verb|\frac{nom}{denom}| & Fraction \\
\verb|\sqrt| & square root \\
\verb|\sqrt[n]| & n-th root \\
\verb|{n \choose k}| & binomial coefficient with braces \\
\verb|{n \atop k}| & binomial coefficient without braces \\
\verb|\begin{array}[...]...\end{array}| & Array environment. Usage similar to tabular. Usefull for matrixes, 
\end{tabular} 

Stuff over/under expressions.

\begin{tabular}
\verb|\overline| & \\
\verb|\underline| & \\
\verb|\overbrace| & \\
\verb|\underbrace| & \\
\verb|\widetilde| & \\
\verb|\widehat| & \\
\verb|\vec| & \\
\verb|\overleftarrow| & \\
\verb|\overrightarrow| & \\
\end{tabular}



\end{document}


